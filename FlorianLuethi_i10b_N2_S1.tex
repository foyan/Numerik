% !TEX TS-program = pdflatex
% !TEX encoding = UTF-8 Unicode

% This is a simple template for a LaTeX document using the "article" class.
% See "book", "report", "letter" for other types of document.

\documentclass[11pt]{article} % use larger type; default would be 10pt

\usepackage[utf8]{inputenc} % set input encoding (not needed with XeLaTeX)

%%% Examples of Article customizations
% These packages are optional, depending whether you want the features they provide.
% See the LaTeX Companion or other references for full information.

%%% PAGE DIMENSIONS
\usepackage{geometry} % to change the page dimensions
\geometry{a4paper} % or letterpaper (US) or a5paper or....
% \geometry{margins=2in} % for example, change the margins to 2 inches all round
% \geometry{landscape} % set up the page for landscape
%   read geometry.pdf for detailed page layout information

\usepackage{graphicx} % support the \includegraphics command and options

% \usepackage[parfill]{parskip} % Activate to begin paragraphs with an empty line rather than an indent

%%% PACKAGES
\usepackage{booktabs} % for much better looking tables
\usepackage{array} % for better arrays (eg matrices) in maths
\usepackage{paralist} % very flexible & customisable lists (eg. enumerate/itemize, etc.)
\usepackage{verbatim} % adds environment for commenting out blocks of text & for better verbatim
\usepackage{subfig} % make it possible to include more than one captioned figure/table in a single float
\usepackage{amsmath}
\usepackage{amssymb}
% These packages are all incorporated in the memoir class to one degree or another...

%%% HEADERS & FOOTERS
\usepackage{fancyhdr} % This should be set AFTER setting up the page geometry
\pagestyle{fancy} % options: empty , plain , fancy
\renewcommand{\headrulewidth}{0pt} % customise the layout...
\lhead{}\chead{}\rhead{}
\lfoot{}\cfoot{\thepage}\rfoot{}

%%% SECTION TITLE APPEARANCE
\usepackage{sectsty}
\allsectionsfont{\sffamily\mdseries\upshape} % (See the fntguide.pdf for font help)
% (This matches ConTeXt defaults)

%%% ToC (table of contents) APPEARANCE
\usepackage[nottoc,notlof,notlot]{tocbibind} % Put the bibliography in the ToC
\usepackage[titles,subfigure]{tocloft} % Alter the style of the Table of Contents
\renewcommand{\cftsecfont}{\rmfamily\mdseries\upshape}
\renewcommand{\cftsecpagefont}{\rmfamily\mdseries\upshape} % No bold!

%%% END Article customizations

%%% The "real" document content comes below...

\title{Numerik 2 - Übung 1}
\author{Florian Lüthi, i10b}
%\date{} % Activate to display a given date or no date (if empty),
         % otherwise the current date is printed 

\begin{document}
\maketitle

\section*{Aufgabe 1}

\begin{enumerate}[a)]
\item $ v(u) = \alpha u + \beta $

\item
$ b = \begin{pmatrix} 0.95 \\ 2.14 \\ 2.98  \end{pmatrix} $, $ x = \begin{pmatrix} \alpha \\ \beta \end{pmatrix} $, $ A = \begin{pmatrix} 1 & 1 \\ 2 & 1 \\ 3 & 1 \end{pmatrix} $

\item durch Normalgleichung:
\begin{eqnarray*}
A^TA  &=& \begin{pmatrix}1 & 2 & 3 \\ 1 & 1 & 1\end{pmatrix} \begin{pmatrix} 1 & 1 \\ 2 & 1 \\ 3 & 1 \end{pmatrix} = \begin{pmatrix} 14 & 6 \\ 6 & 3 \end{pmatrix} \\
A^Tb &=& \begin{pmatrix}1 & 2 & 3 \\ 1 & 1 & 1\end{pmatrix} \begin{pmatrix} 0.95 \\ 2.14 \\ 2.98 \end{pmatrix} = \begin{pmatrix} 14.17 \\ 6.07\end{pmatrix} \\
\Rightarrow \begin{pmatrix} \alpha^* \\ \beta^* \end{pmatrix} &=& \begin{pmatrix} 14.17 \\ 6.07 \end{pmatrix} \begin{pmatrix} 14 & 6 \\ 6 & 3 \end{pmatrix}^{-1} = \begin{pmatrix} 1.0150 \\ -0.0067 \end{pmatrix} \\
\Rightarrow v(u) &=& 1.015u - 0.0067
\end{eqnarray*}

oder durch analytische Minimierung der Abweichung:

\begin{eqnarray*}
F(\alpha, \beta) &=& (-0.95 + \alpha + \beta)^2 + (-2.14 + 2\alpha + \beta)^2 + (-2.98 + 3\alpha + \beta)^2 \\
\Rightarrow \frac{\partial F}{\partial \alpha} &=& -1.9 + 2\alpha + 2\beta - 8.56 + 8\alpha + 4\beta - 17.88 + 18\alpha + 6\beta \overset{!}{=} 0 \\
\frac{\partial F}{\partial \beta} &=& -1.9 + 2\alpha + 2\beta - 4.28 + 4\alpha + 2\beta - 5.96 + 6\alpha + 2\beta \overset{!}{=} 0 \\
\Rightarrow \beta &=& \frac{-28 \alpha +28.34}{12} \\
\Rightarrow \alpha &=& \frac{-6\beta + 12.14}{12} = \frac{14 \alpha -2.03}{12} \\
\Rightarrow 12\alpha &=& 14\alpha - 2.03 \Rightarrow \alpha = 1.015 \\
\Rightarrow \beta &=& \frac{-28 \cdot 1.015 +28.34}{12} = -0.0067 \\
\Rightarrow v(u) &=& 1.015u - 0.0067
\end{eqnarray*}

\item Bitteschön. Rot sind die verbundenen Messpunkte, grün die Ausgleichsgerade.

\end{enumerate}


\includegraphics{N2_S1_Aufg1d.png}

\section*{Aufgabe 2}

\begin{enumerate}[a)]

\item
\begin{eqnarray*}
F(\alpha, \beta) = (a_{11}\alpha + a_{12}\beta - b_1)^2+(a_{21}\alpha + a_{22}\beta - b_2)^2+(a_{31}\alpha + a_{32}\beta - b_3)^2
\end{eqnarray*}

\item
\begin{eqnarray*}
\frac{\partial F}{\partial\alpha} &=& 2a_{11}(a_{11}\alpha + a_{12}\beta - b_1) + 2a_{21}(a_{21}\alpha + a_{22}\beta - b_2) + 2a_{31}(a_{31}\alpha + a_{32}\beta - b_3) \\
\frac{\partial F}{\partial\beta} &=& 2a_{12}(a_{11}\alpha + a_{12}\beta - b_1) + 2a_{22}(a_{21}\alpha + a_{22}\beta - b_2) + 2a_{32}(a_{31}\alpha + a_{32}\beta - b_3) \\
\Rightarrow \operatorname{grad}(F) &=& \begin{pmatrix}
2a_{11}(a_{11}\alpha + a_{12}\beta - b_1) + 2a_{21}(a_{21}\alpha + a_{22}\beta - b_2) + 2a_{31}(a_{31}\alpha + a_{32}\beta - b_3) \\
2a_{12}(a_{11}\alpha + a_{12}\beta - b_1) + 2a_{22}(a_{21}\alpha + a_{22}\beta - b_2) + 2a_{32}(a_{31}\alpha + a_{32}\beta - b_3)
\end{pmatrix}
\end{eqnarray*}

\item Gesucht: $A^TAx = A^Tb$
\begin{eqnarray*}
&&\begin{pmatrix} 2a_{11}(a_{11}\alpha + a_{12}\beta - b_1) \\ 2a_{12}(a_{11}\alpha + a_{12}\beta - b_1)
\end{pmatrix} + 
\begin{pmatrix} 2a_{21}(a_{21}\alpha + a_{22}\beta - b_2) \\  2a_{22}(a_{21}\alpha + a_{22}\beta - b_2)
\end{pmatrix} + 
\begin{pmatrix} 2a_{31}(a_{31}\alpha + a_{32}\beta - b_3) \\ 2a_{32}(a_{31}\alpha + a_{32}\beta - b_3)
\end{pmatrix} = \begin{pmatrix}0 \\ 0 \end{pmatrix} \\
&& 2\begin{pmatrix}a_{11}\\a_{12}\end{pmatrix}(a_{11}\alpha + a_{12}\beta - b_1)
+ 2\begin{pmatrix}a_{21}\\a_{22}\end{pmatrix}(a_{21}\alpha + a_{22}\beta - b_2)
+ 2\begin{pmatrix}a_{31}\\a_{32}\end{pmatrix}(a_{31}\alpha + a_{32}\beta - b_3)
= \begin{pmatrix}0 \\ 0 \end{pmatrix} \\
&& \begin{pmatrix}a_{11}\\a_{12}\end{pmatrix}(a_{11}\alpha + a_{12}\beta - b_1)
+ \begin{pmatrix}a_{21}\\a_{22}\end{pmatrix}(a_{21}\alpha + a_{22}\beta - b_2)
+ \begin{pmatrix}a_{31}\\a_{32}\end{pmatrix}(a_{31}\alpha + a_{32}\beta - b_3)
= \begin{pmatrix}0 \\ 0 \end{pmatrix} \\
&& \begin{pmatrix}a_{11} & a_{21} & a_{31} \\ a_{12} & a_{22} & a_{32}\end{pmatrix}
\begin{pmatrix} a_{11}\alpha + a_{12}\beta - b_1 \\ a_{21}\alpha + a_{22}\beta - b_2 \\ a_{31}\alpha + a_{32}\beta - b_3 \end{pmatrix} = \begin{pmatrix}0\\0\end{pmatrix} \\
&& \begin{pmatrix}a_{11} & a_{21} & a_{31} \\ a_{12} & a_{22} & a_{32}\end{pmatrix}
\begin{pmatrix} a_{11}\alpha + a_{12}\beta \\ a_{21}\alpha + a_{22}\beta \\ a_{31}\alpha + a_{32}\beta \end{pmatrix}
- \begin{pmatrix}a_{11} & a_{21} & a_{31} \\ a_{12} & a_{22} & a_{32}\end{pmatrix}
\begin{pmatrix} b_1 \\  b_2 \\ b_3 \end{pmatrix}
= \begin{pmatrix}0\\0\end{pmatrix} \\
&& \begin{pmatrix}a_{11} & a_{21} & a_{31} \\ a_{12} & a_{22} & a_{32}\end{pmatrix}
\begin{pmatrix} a_{11}\alpha + a_{12}\beta \\ a_{21}\alpha + a_{22}\beta \\ a_{31}\alpha + a_{32}\beta \end{pmatrix}
= \begin{pmatrix}a_{11} & a_{21} & a_{31} \\ a_{12} & a_{22} & a_{32}\end{pmatrix}
\begin{pmatrix} b_1 \\  b_2 \\ b_3 \end{pmatrix} \\
&&
\underbrace{\begin{pmatrix}a_{11} & a_{21} & a_{31} \\ a_{12} & a_{22} & a_{32}\end{pmatrix}}_{A^T}
\underbrace{\begin{pmatrix} a_{11} & a_{12} \\ a_{21} & a_{22} \\ a_{31} & a_{32} \end{pmatrix}}_{A}
\underbrace{\begin{pmatrix} \alpha \\ \beta \end{pmatrix}}_{x}
=
\underbrace{\begin{pmatrix}a_{11} & a_{21} & a_{31} \\ a_{12} & a_{22} & a_{32}\end{pmatrix}}_{A^T}
\underbrace{\begin{pmatrix} b_1 \\  b_2 \\ b_3 \end{pmatrix}}_{b}
\end{eqnarray*} 

qed.

\end{enumerate}

\section*{Aufgabe 3}

\begin{enumerate}[a)]

\item Ansatz: $v(u) = \alpha u + \beta $

\begin{eqnarray*}
A = \begin{pmatrix} 0 & 1 \\ 1 & 1 \\ 2 & 1 \\ 3 & 1 \\ \vdots & \vdots \end{pmatrix},
b = \begin{pmatrix} -3 \\ -6 \\ -9 \\ -12 \\ \vdots \end{pmatrix}, x = \begin{pmatrix}\alpha \\ \beta\end{pmatrix}
\end{eqnarray*}
\begin{eqnarray*}
A^TA = \begin{pmatrix} 385 & 55 \\ 55 & 11 \end{pmatrix}, A^Tb = \begin{pmatrix}1554 \\ 126 \end{pmatrix}
\end{eqnarray*}
\begin{eqnarray*}
\alpha^* = 8.4, \beta^* = -30.5455  \\ \Rightarrow v(u) = 8.4u - 30.5455
\end{eqnarray*}

\item Ansatz: $v(u) = \alpha u^2 + \beta u + \gamma $

\begin{eqnarray*}
A = \begin{pmatrix} 0 & 0 & 1 \\ 1 & 1 & 1 \\ 4 & 2 & 1 \\ 9 & 3 & 1 \\ \vdots & \vdots & \vdots \end{pmatrix},
b = \begin{pmatrix} -3 \\ -6 \\ -9 \\ -12 \\ \vdots \end{pmatrix}, x = \begin{pmatrix}\alpha \\ \beta \\ \gamma \end{pmatrix}
\end{eqnarray*}
\begin{eqnarray*}
A^TA = \begin{pmatrix} 25333 & 3025 & 385 \\ 3025 & 385 & 55 \\ 385 & 55 & 11\end{pmatrix}, A^Tb = \begin{pmatrix} 15782 \\ 1554 \\ 126 \end{pmatrix}
\end{eqnarray*}
\begin{eqnarray*}
\alpha^* =  2.4848, \beta^* =  -16.4485, \gamma^* = 6.7273 \\ \Rightarrow v(u) = 2.4848u^2 - 16.4485u + 6.7273
\end{eqnarray*}

\item Ansatz: $v(u) = \alpha u^3 + \beta u^2 + \gamma u + \delta $

\begin{eqnarray*}
A = \begin{pmatrix} 0& 0 & 0 & 1 \\ 1 & 1 & 1 & 1 \\ 8 & 4 & 2 & 1 \\ 27 & 9 & 3 & 1 \\ \vdots & \vdots & \vdots & \vdots \end{pmatrix},
b = \begin{pmatrix} -3 \\ -6 \\ -9 \\ -12 \\ \vdots \end{pmatrix}, x = \begin{pmatrix}\alpha \\ \beta \\ \gamma \\ \delta \end{pmatrix}
\end{eqnarray*}
\begin{eqnarray*}
A^TA = \begin{pmatrix}  1978405  &    220825   &    25333      &  3025 \\
      220825  &     25333 &       3025   &      385\\
       25333  &      3025  &       385     &     55\\
        3025   &      385    &      55   &       11
\end{pmatrix}, A^Tb = \begin{pmatrix} 154116 \\ 15782 \\ 1554 \\ 126 \end{pmatrix}
\end{eqnarray*}
\begin{eqnarray*}
\alpha^* =  0.2815, \beta^* =   -1.7372, \gamma^* =  -0.3485, \delta^* = -3.4056  \\ \Rightarrow v(u) = 0.2815u^3  -1.7372u^2 -  0.3485u -3.4056
\end{eqnarray*}

\item Die dunkelblaue Kurve zeigt die Messwerte, die grüne Kurve den linearen Ansatz, die rote den quadratischen und die hellblaue den kubischen:

\includegraphics{N2_S1_Aufg3d.png}

\item Dass der lineare Ansatz nicht passen würde, war vorauszusehen. Dass hingegen der kubische die Messreihe ziemlich genau abbildet ist schon erstaunlich.

\end{enumerate}

\end{document}
