% !TEX TS-program = pdflatex
% !TEX encoding = UTF-8 Unicode

% This is a simple template for a LaTeX document using the "article" class.
% See "book", "report", "letter" for other types of document.

\documentclass[11pt]{article} % use larger type; default would be 10pt

\usepackage[utf8]{inputenc} % set input encoding (not needed with XeLaTeX)

%%% Examples of Article customizations
% These packages are optional, depending whether you want the features they provide.
% See the LaTeX Companion or other references for full information.

%%% PAGE DIMENSIONS
\usepackage{geometry} % to change the page dimensions
\geometry{a4paper} % or letterpaper (US) or a5paper or....
% \geometry{margins=2in} % for example, change the margins to 2 inches all round
% \geometry{landscape} % set up the page for landscape
%   read geometry.pdf for detailed page layout information

\usepackage{graphicx} % support the \includegraphics command and options

% \usepackage[parfill]{parskip} % Activate to begin paragraphs with an empty line rather than an indent

%%% PACKAGES
\usepackage{booktabs} % for much better looking tables
\usepackage{array} % for better arrays (eg matrices) in maths
\usepackage{paralist} % very flexible & customisable lists (eg. enumerate/itemize, etc.)
\usepackage{verbatim} % adds environment for commenting out blocks of text & for better verbatim
\usepackage{subfig} % make it possible to include more than one captioned figure/table in a single float
\usepackage{amsmath}
\usepackage{amssymb}
% These packages are all incorporated in the memoir class to one degree or another...

%%% HEADERS & FOOTERS
\usepackage{fancyhdr} % This should be set AFTER setting up the page geometry
\pagestyle{fancy} % options: empty , plain , fancy
\renewcommand{\headrulewidth}{0pt} % customise the layout...
\lhead{}\chead{}\rhead{}
\lfoot{}\cfoot{\thepage}\rfoot{}

%%% SECTION TITLE APPEARANCE
\usepackage{sectsty}
\allsectionsfont{\sffamily\mdseries\upshape} % (See the fntguide.pdf for font help)
% (This matches ConTeXt defaults)

%%% ToC (table of contents) APPEARANCE
\usepackage[nottoc,notlof,notlot]{tocbibind} % Put the bibliography in the ToC
\usepackage[titles,subfigure]{tocloft} % Alter the style of the Table of Contents
\renewcommand{\cftsecfont}{\rmfamily\mdseries\upshape}
\renewcommand{\cftsecpagefont}{\rmfamily\mdseries\upshape} % No bold!

%%% END Article customizations

%%% The "real" document content comes below...

\title{Numerik 2 - Übung 2}
\author{Florian Lüthi, i10b}
%\date{} % Activate to display a given date or no date (if empty),
         % otherwise the current date is printed 

\begin{document}
\maketitle

\section*{Aufgabe 1}

\begin{enumerate}[a)]

\item %a
Der gewünschte Plot.

\includegraphics{N2_S2_Aufg1a.png}

\item %b
Die logarithmierten Werte.

\begin{tabular}{r|r|r|r}
$u$ & $v$ & $log_{10}u$ & $log_{10}v$ \\
\hline
57.9 &	88 &		1.76267856372744 &	1.94448267215017 \\
108.2 &	225 &	2.03422726077055 &	2.35218251811136 \\
149.6 &	365 &	2.17493159352844 &	2.56229286445647 \\
227.9 &	687 &	2.35774432518038 &	2.83695673705955 \\
778.1 &	4329 &	2.89103541531531 &	3.63638758581316 \\
1428.2 &	10753 &	3.15478902873875 &	4.03152964580342 \\
2837.9 &	30660 &	3.45299708801598 &	4.48657215051836 \\
4488.9 &	60150 &	3.65213993065629 &	4.77923563167586 \\
5876.7 &	90670 &	3.76913352095601 &	4.95746361572993 \\

\end{tabular}

\includegraphics{N2_S2_Aufg1b.png}

\item %c

\begin{eqnarray*}
\log_{10}{v} &=& \log_{10}(\alpha u^\beta) \\
\log_{10}{v} &=& \log_{10}\alpha + \beta \log_{10}u
\end{eqnarray*}

\item Ansatz:
\begin{eqnarray*}
\tilde{v} = \beta\tilde{u} + \tilde{\alpha} 
\end{eqnarray*}
\begin{eqnarray*}
A = \begin{pmatrix}
1.7626\dots & 1 \\
2.0342\dots & 1 \\
2.1749\dots & 1 \\
\vdots & \vdots
\end{pmatrix},
b = \begin{pmatrix}
1.9444 \\ 2.3521 \\ 2.5622 \\ \vdots
\end{pmatrix},
x = \begin{pmatrix}\beta \\ \tilde{\alpha} \end{pmatrix}
\end{eqnarray*}
\begin{eqnarray*}
A^TA = \begin{pmatrix}  75.3129  &  25.2497 \\
   25.2497  &  9.0000 \end{pmatrix}, A^Tb = \begin{pmatrix}  95.3375 \\ 31.5871 \end{pmatrix}
 \Rightarrow \begin{pmatrix} \beta^* \\ \tilde{\alpha}^* \end{pmatrix} = \begin{pmatrix}  1.5017 \\
   -0.7034 \end{pmatrix}
\end{eqnarray*}
\begin{eqnarray*}
\Rightarrow \tilde{v} = 1.5017\tilde{u} - 0.7034
\end{eqnarray*}

\includegraphics{N2_S2_Aufg1d.png}

\item
\begin{eqnarray*}
%\log_{10} v &=& \tilde v \\
%&=& \beta\tilde{u} - \tilde{\alpha} \\
%&=& \beta \cdot \log_{10}u - \tilde{\alpha} \\
%\Rightarrow v &=& 10^{1.5017 u - 0.7034 }
v &=& \alpha u^\beta \\
&=& 10^{\tilde \alpha} u^\beta \\
&=&  0.1980 u^{1.5017}
\end{eqnarray*}

\includegraphics{N2_S2_Aufg1e.png}

\item

Da die Konditionszahl ein Faktor ist, kann Linearität angenommen werden, und aus den gegebenen Fehlern erkennt man folgenden Zusammenhang:
\begin{eqnarray*}
F_i &=& |v_i - \alpha u_i^\beta| = 0.9\cdot v_i \Rightarrow \alpha u_i^\beta = 0.1\cdot v_i \\
\Rightarrow \tilde{F_i} &=& |\log_{10}v_i - \log_{10}(\alpha u_i^\beta)| = \log_{10}\left(\frac{v_i}{\alpha u_i^\beta}\right) \\
&=& log_{10}\left(\frac{v_i}{0.1\cdot v_i}\right) = log_{10}10 = 1
\end{eqnarray*}

Der Fehler der logarithmierten Funktion ist konstant 1.

\item

Für die originale Funktion:
\begin{eqnarray*}
F_{ri} &=& \frac{|0.9\cdot v_i|}{|v_i|} = 0.9 \\
\end{eqnarray*}

Für die logarithmierte Funktion:
\begin{eqnarray*}
\tilde{F_{ri}} &=& \frac{1}{|\log_{10}v_i|} \\
\end{eqnarray*}

\begin{tabular}{r|r|r|r}
$v$ & $log_{10}v$ & $F_ri$ & $\tilde{F_ri}$ \\
\hline
88 &		1.94448267215017 & 0.9 & 0.514275603646404 \\ 
225 &	2.35218251811136& 0.9 & 0.425137076863801  \\
365 &	2.56229286445647 & 0.9 & 0.390275449723865  \\
687 &	2.83695673705955& 0.9 & 0.352490394702487  \\
4329 &	3.63638758581316 & 0.9 & 0.274998188834809  \\
10753 &	4.03152964580342 & 0.9 & 0.248044808759112  \\
30660 &	4.48657215051836 & 0.9 & 0.222887310501507  \\
60150 &	4.77923563167586 & 0.9 & 0.209238480181264  \\
90670 &	4.95746361572993 & 0.9 & 0.201716054320000 \\
\end{tabular}

\item Unter Annahme einer derartigen Fehlerbehaftung ist die Berechnung des logarithmierten Ausgleichsproblems nicht sinnvoll. Es wird nämlich versucht, die quadrierten absoluten Fehler zu minimieren, und die sind ja konstant 1. Ergo kann das Verfahren eine beliebige Ausgleichsgerade ausspucken.

\end{enumerate}

\section*{Aufgabe 2}

\begin{enumerate}[a)]

\item

\begin{eqnarray*}
A &=& \begin{pmatrix} 1 & 0 \\ 1 & 1 \\ 1 & 2 \end{pmatrix} \\
\Rightarrow a_1 &=& \begin{pmatrix} 1 \\ 1 \\ 1 \end{pmatrix}, ||a_1||_2 = \sqrt{3} = 1.7321 \\
\Rightarrow v_1 &=& a_1 + 1.7321e_1 = \begin{pmatrix}2.7321 \\ 1 \\ 1 \end{pmatrix}, Q_{v_1}a_1 = -1.7321e_1 = \begin{pmatrix} -1.7321 \\ 0 \\0 \end{pmatrix}\\
 x_1 &=& \begin{pmatrix} 0 \\ 1\\ 2 \end{pmatrix} \Rightarrow Q_{v_1}x_1 = x_1  - \frac 2 {||v_1||_2^2} (v_1^T x_1)v_1 = x_1 - \frac 2 {9.4641}(v_1^Tx_1)v_1 \\
&=& \begin{pmatrix} 0\\ 1\\ 2 \end{pmatrix} - 0.2113 \cdot 3 v_1 =\begin{pmatrix} 0\\ 1\\ 2 \end{pmatrix}- \begin{pmatrix}  1.7321 \\   0.6340 \\   0.6340 \end{pmatrix} = \begin{pmatrix} -1.7321\\0.3660 \\ 1.3660 \end{pmatrix} \\
\Rightarrow Q_{v_1}A &=& \begin{pmatrix}  -1.7321 & -1.7321 \\ 0 &0.3660 \\ 0 & 1.3660  \end{pmatrix}\\\\
a_2 &=& \begin{pmatrix} 0.3660 \\  1.3660 \end{pmatrix}, ||a_2||_2 = 1.4142\\
\Rightarrow v_2 &=& a_2 + 1.4142e_1 = \begin{pmatrix}  1.7802 \\ 1.3660 \end{pmatrix}, Q_{v_2}a_2 = -1.4142e_1 = \begin{pmatrix}-1.4142\\0\end{pmatrix} \\
\Rightarrow R &=& \begin{pmatrix}  -1.7321 & -1.7321 \\ 0 & -1.4142 \\ 0 & 0   \end{pmatrix}
\end{eqnarray*}

\item

\begin{eqnarray*}
Q_1b &=& b - \frac 2 {||v_1||_2^2}(v_1^T b)v_1 = \begin{pmatrix} 0 \\ 2 \\ 1 \end{pmatrix} - \frac 2 {9.4641}\cdot 3\begin{pmatrix}2.7321 \\ 1 \\ 1 \end{pmatrix} = \begin{pmatrix}-1.7321 \\ 1.3660 \\ 0.3660 \end{pmatrix} \\
Q_2Q_1b &=& Q_2\begin{pmatrix}-1.7321 \\ 1.3660 \\ 0.3660 \end{pmatrix} = \begin{pmatrix} -1.7321 \\\hline
\begin{pmatrix} Q_1b_2 \\ Q_1b_3 \end{pmatrix} - \frac 2 {||v_2||_2^2}\left(v_2^T \begin{pmatrix} Q_1b_2 \\ Q_1b_3 \end{pmatrix}\right)v_2  \end{pmatrix} \\
&=& \begin{pmatrix} -1.7321 \\\hline
\begin{pmatrix}1.3660 \\ 0.3660 \end{pmatrix} - \frac 2 {5.0351}\cdot 2.9317 \begin{pmatrix}  1.7802 \\ 1.3660 \end{pmatrix}  \end{pmatrix}
=  \begin{pmatrix}  -1.7321 \\  -0.7071 \\ -1.2247 \end{pmatrix} \\
\end{eqnarray*}

\begin{eqnarray*}
\begin{pmatrix} -1.7321 & -1.7321 \\  0 & -1.4142 \end{pmatrix} x^* &=& \begin{pmatrix} -1.7321 \\ -0.7071 \end{pmatrix} \\
\Rightarrow x_1^* &=& 0.5 \\  x_2^* &=& 0.5 \\
v(u) &=& 0.5u + 0.5
\end{eqnarray*}

Test mit Normalgleichung:

\begin{eqnarray*}
A^TA = \begin{pmatrix} 3 & 3 \\ 3  & 5\end{pmatrix}, A^T b = \begin{pmatrix} 3 \\ 4 \end{pmatrix} \Rightarrow x^* = \begin{pmatrix} 0.5 \\ 0.5 \end{pmatrix}
\end{eqnarray*}

Und weil's so lange gedauert hat weil ich mich 1000 mal verrechnet habe, möchte ich es jetzt wirklich auskosten und mache noch ein Bildchen.

\includegraphics{N2_S2_Aufg2b.png}

\end{enumerate}


\end{document}
